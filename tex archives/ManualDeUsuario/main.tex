\documentclass[11pt, a4paper]{article}

\usepackage[utf8]{inputenc}
\usepackage[T1]{fontenc}
\usepackage[spanish]{babel}
\usepackage{graphicx}
\usepackage{geometry}
\geometry{left=2.5cm, right=2.5cm, top=2.5cm, bottom=2.5cm}
\usepackage{hyperref}
\usepackage{parskip}
\usepackage{amsmath} % Paquete matemático
\usepackage{listings} % Para bloques de código
\usepackage{xcolor} % Para colores en el código
\usepackage{caption} % Para personalizar captiones

\usepackage{fontspec}
\setmainfont{Source Sans Pro}
\setsansfont{Source Sans Pro}
\setmonofont{Source Code Pro}[Scale=MatchLowercase]

% Configuración de colores para listings (igual a tu configuración original)
\definecolor{codebg}{rgb}{0.95,0.95,0.95}
\definecolor{codeframe}{rgb}{0.82,0.82,0.82}
\definecolor{keyword}{rgb}{0.0,0.2,0.65}
\definecolor{comment}{rgb}{0.25,0.5,0.35}
\definecolor{string}{rgb}{0.58,0.0,0.05}
\definecolor{linenumber}{rgb}{0.45,0.45,0.45}

\lstdefinestyle{mystyle}{
  backgroundcolor=\color{codebg},
  frame=single,
  rulecolor=\color{codeframe},
  framesep=6pt,
  framerule=0.6pt,
  basicstyle=\ttfamily\small,
  keywordstyle=\color{keyword}\bfseries,
  commentstyle=\itshape\color{comment},
  stringstyle=\color{string},
  numbers=left,
  numberstyle=\tiny\color{linenumber},
  stepnumber=1,
  numbersep=8pt,
  showstringspaces=false,
  breaklines=true,
  postbreak=\mbox{\textcolor{codeframe}{$\hookrightarrow$}\space},
  tabsize=2,
  captionpos=b,
  xleftmargin=6pt,
  xrightmargin=0pt
}

\lstset{style=mystyle, language=Python}

\title{\textbf{Prueba de Rangos con Signo de Wilcoxon para Observaciones Pareadas} \\ \large Proyecto Final de Probabilidad y Estadística}
\author{%
Álvarez Ortiz Arley Santiago -- 20241020008 \\
Martínez Pardo Silvana -- 20241020010 \\
Moreno Granado Sergio Leonardo -- 20242020091 \\
Rodríguez Camacho Juan Esteban -- 20241020029 \\[1em]
\emph{Docente:} Diego Alberto Chitiva Huertas
}
\date{Semestre 2025-3 \\ Diciembre de 2025, Bogotá D.C.}

% --- PAQUETES AUXILIARES ---
\usepackage{caption}
\usepackage[style=apa, backend=biber]{biblatex}
\addbibresource{references.bib}


\begin{document}

% Portada con logo (comenta la siguiente línea si no tienes la imagen)
\begin{titlepage}
    \centering
    % Si tienes la imagen UDLogo.png, descomenta la línea siguiente
    \includegraphics[width=6cm]{figures/UDLogo.png}\\[1cm]
    
    {\LARGE Universidad Distrital Francisco José de Caldas\par}
    {\large Facultad de Ingeniería\par}
    {\large Ingeniería de Sistemas\par}
    \vspace{2cm}
    
    {\huge\bfseries Manual de Usuario\par}
    {\huge\bfseries Blog \textit{Odally}\par}
    \vspace{1.5cm}
    
    {\large
    \begin{tabular}{l}
    Alejandra Munevar -- 2024 \\
    Dylan Silva -- 2024 \\
    Moreno Granado Sergio Leonardo -- 20242020091
    \end{tabular}\par}
    \vspace{1.5cm}
    
    {\large \emph{Docente:} Lilia Marcela Espinoza Rodríguez\par}
    \vspace{2cm}
    
    {\large Proyecto Final de Programación Avanzada\par}
    {\large Semestre 2025-3\par}
    \vfill
    
    {\large Diciembre de 2025, Bogotá D.C.\par}
\end{titlepage}

\tableofcontents
\newpage

\section{Introducción}
Este manual tiene como objetivo guiar al usuario en el uso del Sistema de Gestión de Contenidos (Blog). La aplicación permite a los usuarios leer artículos publicados y a los administradores gestionar dicho contenido.

\section{Acceso a la Aplicación}
Para acceder al sistema, abra su navegador web favorito e ingrese la siguiente dirección URL:
\begin{center}
    \texttt{http://localhost:8080/AdvancedFinalProject/}
\end{center}

\section{Área Pública}
Esta sección es accesible para cualquier visitante y no requiere autenticación.

\subsection{Ver Lista de Artículos}
Al ingresar a la aplicación, verá la página principal con un listado de los artículos más recientes.
\begin{itemize}
    \item Cada tarjeta muestra el título, un resumen del contenido, el autor y la fecha de publicación.
\end{itemize}

\subsection{Leer un Artículo}
Para leer el contenido completo de un artículo:
\begin{enumerate}
    \item Haga clic en el botón \textbf{"Leer más"} ubicado en la parte inferior de la tarjeta del artículo.
    \item Se abrirá una nueva página con el texto completo.
    \item Para volver, puede usar el botón \textbf{"Volver al inicio"} o el menú de navegación.
\end{enumerate}

\section{Área de Administración}
Esta sección es exclusiva para usuarios autorizados.

\subsection{Iniciar Sesión}
1. En la barra de navegación superior, haga clic en \textbf{"Iniciar Sesión"}.
2. Ingrese su nombre de usuario (ej. \texttt{admin}) y contraseña.
3. Haga clic en el botón \textbf{"Ingresar"}.

\subsection{Dashboard}
Una vez autenticado, será redirigido al Dashboard, donde podrá ver estadísticas rápidas sobre el número de artículos publicados.

\subsection{Gestión de Artículos}
Desde el menú de administración, seleccione \textbf{"Artículos"} para acceder a las opciones de gestión.

\subsubsection{Crear Nuevo Artículo}
1. Haga clic en el botón verde \textbf{"Nuevo Artículo"}.
2. Complete el formulario con el \textbf{Título} y el \textbf{Contenido}.
3. Haga clic en \textbf{"Guardar"}.

\subsubsection{Editar Artículo}
1. En la lista de artículos, identifique el que desea modificar.
2. Haga clic en el botón azul \textbf{"Editar"}.
3. Modifique la información necesaria y haga clic en \textbf{"Actualizar"}.

\subsubsection{Eliminar Artículo}
1. En la lista de artículos, identifique el que desea borrar.
2. Haga clic en el botón rojo \textbf{"Eliminar"}.
3. Confirme la acción si el sistema lo solicita.

\subsection{Cerrar Sesión}
Para salir del sistema de forma segura, haga clic en el botón \textbf{"Cerrar Sesión"} ubicado en la esquina superior derecha de la barra de navegación.

% --- BIBLIOGRAFÍA ---
\printbibliography[title={Bibliografía}]

\end{document}