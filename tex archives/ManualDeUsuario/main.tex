\documentclass[11pt, a4paper]{article}

\usepackage[utf8]{inputenc}
\usepackage[T1]{fontenc}
\usepackage[spanish]{babel}
\usepackage{graphicx}
\usepackage{geometry}
\geometry{left=2.5cm, right=2.5cm, top=2.5cm, bottom=2.5cm}
\usepackage{hyperref}
\usepackage{parskip}
\usepackage{amsmath}
\usepackage{listings}
\usepackage{xcolor}
\usepackage{caption}

% IMPORTANTE: Este documento requiere XeLaTeX para compilar debido al uso de fuentes personalizadas
% Usar: xelatex main.tex
\usepackage{fontspec}
\setmainfont{Source Sans Pro}
\setsansfont{Source Sans Pro}
\setmonofont{Source Code Pro}[Scale=MatchLowercase]

% Configuración de colores para listings
\definecolor{codebg}{rgb}{0.95,0.95,0.95}
\definecolor{codeframe}{rgb}{0.82,0.82,0.82}
\definecolor{keyword}{rgb}{0.0,0.2,0.65}
\definecolor{comment}{rgb}{0.25,0.5,0.35}
\definecolor{string}{rgb}{0.58,0.0,0.05}
\definecolor{linenumber}{rgb}{0.45,0.45,0.45}

\lstdefinestyle{mystyle}{
  backgroundcolor=\color{codebg},
  frame=single,
  rulecolor=\color{codeframe},
  framesep=6pt,
  framerule=0.6pt,
  basicstyle=\ttfamily\small,
  keywordstyle=\color{keyword}\bfseries,
  commentstyle=\itshape\color{comment},
  stringstyle=\color{string},
  numbers=left,
  numberstyle=\tiny\color{linenumber},
  stepnumber=1,
  numbersep=8pt,
  showstringspaces=false,
  breaklines=true,
  postbreak=\mbox{\textcolor{codeframe}{$\hookrightarrow$}\space},
  tabsize=2,
  captionpos=b,
  xleftmargin=6pt,
  xrightmargin=0pt
}

\lstset{style=mystyle, language=Java}

\begin{document}

% Portada con logo
\begin{titlepage}
    \centering
    % Si tienes la imagen UDLogo.png, descomenta la línea siguiente
    % \includegraphics[width=6cm]{figures/UDLogo.png}\\[1cm]
    
    {\LARGE Universidad Distrital Francisco José de Caldas\par}
    {\large Facultad de Ingeniería\par}
    {\large Ingeniería de Sistemas\par}
    \vspace{2cm}
    
    {\huge\bfseries Manual de Usuario\par}
    {\huge\bfseries Odally | Blog\par}
    \vspace{1.5cm}
    
    {\large
    \begin{tabular}{l}
    Dylan David Silva Orrego -- 20242020130 \\
    Maria Alejandra Munevar Barrera -- 20242020145
    \end{tabular}\par}
    \vspace{1.5cm}
    
    {\large \emph{Profesora:} Lilia Marcela Espinosa Rodríguez\par}
    \vspace{2cm}
    
    {\large Proyecto Final de Programación Avanzada\par}
    {\large Semestre 2025-3\par}
    \vfill
    
    {\large Diciembre de 2025, Bogotá D.C.\par}
\end{titlepage}

\tableofcontents
\newpage

\section{Introducción}

\subsection{¿Qué es Odally | Blog?}

\textbf{Odally | Blog} es un sistema moderno de gestión de contenidos desarrollado con tecnologías JavaWeb (Servlets/JSP). El proyecto permite a los usuarios visualizar artículos publicados y a los administradores gestionar todo el contenido del blog de manera sencilla e intuitiva.

Este sistema fue desarrollado como proyecto final de la materia Programación Avanzada, aplicando principios SOLID, patrones de diseño y buenas prácticas de ingeniería de software.

\subsection{Propósito del Manual}

Este manual tiene como objetivo guiar a los usuarios finales en el uso del sistema Odally | Blog. Abarca desde la instalación inicial hasta el uso avanzado de las funcionalidades administrativas.

\subsection{Audiencia Objetivo}

Este manual está dirigido a:
\begin{itemize}
    \item \textbf{Visitantes:} Usuarios que desean leer artículos del blog
    \item \textbf{Usuarios Registrados:} Personas con cuenta en el sistema
    \item \textbf{Administradores:} Usuarios con permisos para gestionar contenido y usuarios
    \item \textbf{Desarrolladores:} Personas que desean instalar y configurar el sistema
\end{itemize}

\newpage
\section{Requisitos del Sistema}

Para ejecutar Odally | Blog, su sistema debe cumplir con los siguientes requisitos:

\subsection{Software Necesario}

\begin{itemize}
    \item \textbf{Java Development Kit (JDK) 17 o superior}
    \begin{itemize}
        \item Descargar de: \url{https://www.oracle.com/java/technologies/downloads/}
        \item Verificar instalación: \texttt{java -version}
    \end{itemize}
    
    \item \textbf{Apache Tomcat 10 o superior}
    \begin{itemize}
        \item Descargar de: \url{https://tomcat.apache.org/download-10.cgi}
        \item Compatible con Jakarta EE 10
    \end{itemize}
    
    \item \textbf{MySQL Server 8.0 o superior}
    \begin{itemize}
        \item Descargar de: \url{https://dev.mysql.com/downloads/mysql/}
        \item Alternativa: XAMPP (\url{https://www.apachefriends.org/})
    \end{itemize}
\end{itemize}

\subsection{Navegadores Compatibles}

El sistema es compatible con las versiones más recientes de:
\begin{itemize}
    \item Google Chrome (versión 90+)
    \item Mozilla Firefox (versión 88+)
    \item Microsoft Edge (versión 90+)
    \item Safari (versión 14+)
\end{itemize}

\subsection{Requisitos de Hardware Mínimos}

\begin{itemize}
    \item \textbf{Procesador:} Intel Core i3 o equivalente
    \item \textbf{Memoria RAM:} 4 GB mínimo (8 GB recomendado)
    \item \textbf{Espacio en Disco:} 500 MB libres
    \item \textbf{Conexión a Internet:} Necesaria para descarga inicial de dependencias
\end{itemize}

\newpage
\section{Instalación}

\subsection{Paso 1: Clonar el Repositorio}

Abra una terminal o símbolo del sistema y ejecute:

\begin{lstlisting}[language=bash]
git clone https://github.com/gefidu/ProyectoFinalAvanzada.git
cd ProyectoFinalAvanzada/JavaWebBlog
\end{lstlisting}

\subsection{Paso 2: Configurar MySQL}

\subsubsection{Iniciar el Servidor MySQL}

\textbf{En Windows con XAMPP:}
\begin{enumerate}
    \item Abrir el Panel de Control de XAMPP
    \item Hacer clic en ``Start'' junto a MySQL
    \item Verificar que el estado sea ``Running''
\end{enumerate}

\textbf{En Linux:}
\begin{lstlisting}[language=bash]
sudo systemctl start mysql
sudo systemctl status mysql
\end{lstlisting}

\subsubsection{Crear la Base de Datos}

Opción 1 - Usando la línea de comandos:
\begin{lstlisting}[language=bash]
mysql -u root -p < setup_database.sql
\end{lstlisting}

Opción 2 - Usando phpMyAdmin:
\begin{enumerate}
    \item Abrir \url{http://localhost/phpmyadmin}
    \item Hacer clic en ``Nuevo'' para crear una base de datos
    \item Nombrarla ``blog\_db''
    \item Ir a la pestaña ``Importar''
    \item Seleccionar el archivo \texttt{setup\_database.sql}
    \item Hacer clic en ``Continuar''
\end{enumerate}

\subsection{Paso 3: Configurar Tomcat en NetBeans}

\begin{enumerate}
    \item Abrir Apache NetBeans
    \item Ir a \texttt{Tools → Servers}
    \item Hacer clic en ``Add Server''
    \item Seleccionar ``Apache Tomcat or TomEE''
    \item Especificar la ruta de instalación de Tomcat
    \item Hacer clic en ``Finish''
\end{enumerate}

\subsection{Paso 4: Importar el Proyecto}

\begin{enumerate}
    \item En NetBeans, ir a \texttt{File → Open Project}
    \item Navegar hasta la carpeta \texttt{JavaWebBlog}
    \item Seleccionar el proyecto y hacer clic en ``Open Project''
    \item Hacer clic derecho en el proyecto → ``Properties''
    \item En ``Run'', seleccionar el servidor Tomcat configurado
    \item Hacer clic en ``OK''
\end{enumerate}

\newpage
\section{Configuración Inicial}

\subsection{Página de Setup de Base de Datos}

Odally | Blog incluye una interfaz web para facilitar la configuración de la conexión a MySQL. Esta es la forma más sencilla de configurar el sistema.

\subsubsection{Acceso a la Página de Setup}

Existen dos formas de acceder:

\textbf{Método 1 - Redirección Automática:}
\begin{itemize}
    \item Si el sistema detecta que no puede conectarse a la base de datos, lo redirigirá automáticamente a la página de configuración
\end{itemize}

\textbf{Método 2 - Acceso Manual:}
\begin{itemize}
    \item Navegar directamente a: \url{http://localhost:8080/JavaWebBlog/setup}
\end{itemize}

\subsubsection{Configurar Usuario y Contraseña de MySQL}

En la página de Setup, complete los siguientes campos:

\begin{itemize}
    \item \textbf{Host:} localhost (o la dirección IP de su servidor MySQL)
    \item \textbf{Puerto:} 3306 (puerto por defecto de MySQL)
    \item \textbf{Base de datos:} blog\_db
    \item \textbf{Usuario:} root (o su usuario de MySQL)
    \item \textbf{Contraseña:} su contraseña de MySQL (dejar en blanco si no tiene)
\end{itemize}

\subsubsection{Probar Conexión}

\begin{enumerate}
    \item Una vez completados los campos, hacer clic en el botón \textbf{``Probar Conexión''}
    \item Si la conexión es exitosa, aparecerá un mensaje verde indicando éxito
    \item Si falla, aparecerá un mensaje de error con detalles del problema
\end{enumerate}

\textbf{Errores comunes:}
\begin{itemize}
    \item ``Connection refused'': MySQL no está ejecutándose
    \item ``Access denied'': Usuario o contraseña incorrectos
    \item ``Unknown database'': La base de datos blog\_db no existe
\end{itemize}

\subsubsection{Ejecutar Script de Base de Datos}

Una vez que la conexión sea exitosa:

\begin{enumerate}
    \item Hacer clic en el botón \textbf{``Guardar Configuración''}
    \item El sistema guardará las credenciales en el archivo \texttt{db.properties}
    \item La aplicación estará lista para usar
    \item Será redirigido automáticamente a la página principal
\end{enumerate}

\newpage
\section{Uso del Sistema - Visitantes}

Los visitantes son usuarios que acceden al blog sin necesidad de autenticarse. Pueden realizar las siguientes acciones:

\subsection{Ver Lista de Artículos}

\begin{enumerate}
    \item Abrir el navegador web
    \item Navegar a: \url{http://localhost:8080/JavaWebBlog/articulos}
    \item Se mostrará la página principal con la lista de artículos recientes
\end{enumerate}

Cada tarjeta de artículo muestra:
\begin{itemize}
    \item Título del artículo
    \item Resumen del contenido
    \item Nombre del autor
    \item Fecha de publicación
    \item Botón ``Leer más''
\end{itemize}

\subsection{Leer un Artículo Completo}

\begin{enumerate}
    \item En la lista de artículos, localizar el artículo de interés
    \item Hacer clic en el botón \textbf{``Leer más''}
    \item Se abrirá una nueva página mostrando el contenido completo
    \item Para volver a la lista, hacer clic en \textbf{``Volver al inicio''}
\end{enumerate}

\subsection{Registrarse en el Sistema}

\begin{enumerate}
    \item En la barra de navegación superior, hacer clic en \textbf{``Registrarse''}
    \item Completar el formulario de registro:
    \begin{itemize}
        \item Nombre completo
        \item Correo electrónico
        \item Nombre de usuario
        \item Contraseña
        \item Confirmar contraseña
    \end{itemize}
    \item Hacer clic en el botón \textbf{``Registrarse''}
    \item Si el registro es exitoso, será redirigido a la página de inicio de sesión
\end{enumerate}

\textbf{Nota:} Los usuarios registrados comienzan con el rol de ``autor'', lo que les permite crear y gestionar sus propios artículos.

\newpage
\section{Uso del Sistema - Usuarios Registrados}

Los usuarios registrados tienen acceso a funcionalidades adicionales después de iniciar sesión.

\subsection{Iniciar Sesión}

\begin{enumerate}
    \item En la barra de navegación superior, hacer clic en \textbf{``Iniciar Sesión''}
    \item Ingresar credenciales:
    \begin{itemize}
        \item \textbf{Usuario:} su nombre de usuario
        \item \textbf{Contraseña:} su contraseña
    \end{itemize}
    \item Hacer clic en el botón \textbf{``Ingresar''}
    \item Si las credenciales son correctas, será redirigido al Dashboard
\end{enumerate}

\textbf{Credenciales de prueba:}
\begin{itemize}
    \item Usuario: admin | Contraseña: admin123
    \item Usuario: dylan | Contraseña: admin123
    \item Usuario: alejandra | Contraseña: admin123
\end{itemize}

\subsection{Cerrar Sesión}

\begin{enumerate}
    \item Hacer clic en el botón \textbf{``Cerrar Sesión''} en la esquina superior derecha
    \item Será redirigido a la página de inicio
    \item Su sesión será cerrada de forma segura
\end{enumerate}

\newpage
\section{Uso del Sistema - Administradores}

Los administradores tienen acceso completo a todas las funcionalidades del sistema.

\subsection{Acceder al Dashboard}

Una vez autenticado como administrador:
\begin{enumerate}
    \item El sistema lo redirige automáticamente al Dashboard
    \item También puede acceder manualmente a: \url{http://localhost:8080/JavaWebBlog/admin/articulos?action=dashboard}
\end{enumerate}

El Dashboard muestra:
\begin{itemize}
    \item Número total de artículos publicados
    \item Acceso rápido a gestión de artículos
    \item Acceso rápido a gestión de usuarios
\end{itemize}

\subsection{Crear Nuevo Artículo}

\begin{enumerate}
    \item En el Dashboard o en la lista de artículos, hacer clic en el botón verde \textbf{``Nuevo Artículo''}
    \item Completar el formulario:
    \begin{itemize}
        \item \textbf{Título:} Título del artículo
        \item \textbf{Contenido:} Texto completo del artículo
    \end{itemize}
    \item Hacer clic en \textbf{``Guardar''}
    \item El artículo será creado y aparecerá en la lista
\end{enumerate}

\subsection{Editar Artículo Existente}

\begin{enumerate}
    \item Ir a \texttt{Admin → Artículos}
    \item En la lista de artículos, localizar el artículo a editar
    \item Hacer clic en el botón azul \textbf{``Editar''}
    \item Modificar el título o contenido según sea necesario
    \item Hacer clic en \textbf{``Actualizar''}
    \item Los cambios se guardarán inmediatamente
\end{enumerate}

\subsection{Eliminar Artículo}

\begin{enumerate}
    \item Ir a \texttt{Admin → Artículos}
    \item En la lista de artículos, localizar el artículo a eliminar
    \item Hacer clic en el botón rojo \textbf{``Eliminar''}
    \item Confirmar la eliminación en el cuadro de diálogo
    \item El artículo será eliminado permanentemente
\end{enumerate}

\textbf{Advertencia:} Esta acción no se puede deshacer.

\subsection{Gestionar Usuarios}

\subsubsection{Listar Usuarios}

\begin{enumerate}
    \item En el menú de administración, hacer clic en \textbf{``Usuarios''}
    \item Se mostrará una tabla con todos los usuarios del sistema
    \item La tabla muestra: ID, Nombre, Email, Usuario, Rol
\end{enumerate}

\subsubsection{Cambiar Rol de Usuario}

\begin{enumerate}
    \item En la lista de usuarios, localizar el usuario
    \item En la columna ``Acciones'', hacer clic en:
    \begin{itemize}
        \item \textbf{``Promover a Admin''} para otorgar permisos de administrador
        \item \textbf{``Demover a Autor''} para quitar permisos de administrador
    \end{itemize}
    \item El cambio se aplicará inmediatamente
\end{enumerate}

\textbf{Restricciones:}
\begin{itemize}
    \item No se puede cambiar el rol del usuario ``admin'' principal
    \item No se puede auto-demover su propio rol de administrador
\end{itemize}

\subsubsection{Eliminar Usuario Individual}

\begin{enumerate}
    \item En la lista de usuarios, localizar el usuario a eliminar
    \item Hacer clic en el botón rojo \textbf{``Eliminar''}
    \item Confirmar la eliminación
    \item El usuario será eliminado permanentemente
\end{enumerate}

\textbf{Restricciones:}
\begin{itemize}
    \item No se puede eliminar el usuario ``admin'' principal
    \item No se puede auto-eliminar
\end{itemize}

\subsubsection{Eliminar Todos los Usuarios No-Administradores}

\begin{enumerate}
    \item En la lista de usuarios, hacer clic en el botón \textbf{``Eliminar Todos los Autores''}
    \item Leer cuidadosamente la advertencia
    \item Confirmar la acción
    \item Todos los usuarios con rol ``autor'' serán eliminados
    \item Los administradores no serán afectados
\end{enumerate}

\textbf{Advertencia:} Esta acción es irreversible y puede eliminar múltiples usuarios a la vez.

\newpage
\section{Solución de Problemas}

\subsection{El Puerto 8080 Está Ocupado}

\textbf{Síntomas:}
\begin{itemize}
    \item Error al iniciar Tomcat: ``Address already in use: bind''
    \item No se puede acceder a \url{http://localhost:8080}
\end{itemize}

\subsubsection{Solución con Permisos de Administrador}

\textbf{En Windows:}
\begin{lstlisting}[language=bash]
# 1. Identificar el proceso que usa el puerto
netstat -ano | findstr :8080

# 2. Detener el proceso (reemplazar PID con el número real)
taskkill /PID <PID> /F
\end{lstlisting}

\textbf{En Linux/Mac:}
\begin{lstlisting}[language=bash]
# 1. Identificar el proceso
sudo lsof -i :8080

# 2. Detener el proceso
sudo kill -9 <PID>
\end{lstlisting}

\subsubsection{Solución SIN Permisos de Administrador (Cambiar Puerto)}

\textbf{Opción 1 - Editar server.xml:}
\begin{enumerate}
    \item Navegar a la carpeta de instalación de Tomcat
    \item Abrir \texttt{conf/server.xml}
    \item Buscar la línea: \texttt{<Connector port="8080" ...}
    \item Cambiar ``8080'' por otro puerto (ej: 8081, 9090)
    \item Guardar y reiniciar Tomcat
    \item Acceder a: \url{http://localhost:NUEVO_PUERTO/JavaWebBlog/}
\end{enumerate}

\textbf{Opción 2 - Desde NetBeans:}
\begin{enumerate}
    \item Click derecho en el proyecto → Properties
    \item Ir a ``Run'' → Hacer clic en el botón ``...'' junto al servidor
    \item Cambiar ``HTTP Port'' a otro puerto disponible
    \item Aplicar cambios y reiniciar
\end{enumerate}

\subsection{No Se Puede Conectar a la Base de Datos}

\textbf{Síntomas:}
\begin{itemize}
    \item Mensaje de error: ``Cannot connect to database''
    \item Redirección automática a la página de Setup
\end{itemize}

\subsubsection{Verificar MySQL Activo}

\textbf{En Windows con XAMPP:}
\begin{enumerate}
    \item Abrir el Panel de Control de XAMPP
    \item Verificar que MySQL esté en estado ``Running''
    \item Si no está activo, hacer clic en ``Start''
\end{enumerate}

\textbf{En Linux:}
\begin{lstlisting}[language=bash]
sudo systemctl status mysql
sudo systemctl start mysql
\end{lstlisting}

\subsubsection{Usar la Página de Setup}

\begin{enumerate}
    \item Navegar a: \url{http://localhost:8080/JavaWebBlog/setup}
    \item Ingresar las credenciales correctas de MySQL
    \item Hacer clic en ``Probar Conexión''
    \item Si es exitoso, hacer clic en ``Guardar Configuración''
\end{enumerate}

\subsection{Error al Cargar Contenido}

\textbf{Síntomas:}
\begin{itemize}
    \item La página carga pero no muestra artículos
    \item Errores en la consola del navegador
\end{itemize}

\subsubsection{Verificar Tablas Creadas}

\begin{lstlisting}[language=sql]
USE blog_db;
SHOW TABLES;

-- Debe mostrar:
-- articulos
-- usuarios

SELECT COUNT(*) FROM articulos;
SELECT COUNT(*) FROM usuarios;
\end{lstlisting}

Si las tablas no existen:
\begin{enumerate}
    \item Ejecutar el script \texttt{setup\_database.sql}
    \item Reiniciar el servidor Tomcat
    \item Intentar nuevamente
\end{enumerate}

\newpage
\section{Preguntas Frecuentes (FAQ)}

\subsection{¿Cómo cambio mi contraseña?}

Actualmente, el sistema no incluye una funcionalidad para que los usuarios cambien su contraseña desde la interfaz web. Un administrador puede actualizar la contraseña directamente en la base de datos ejecutando:

\begin{lstlisting}[language=sql]
UPDATE usuarios 
SET password = SHA2('nueva_contraseña', 256) 
WHERE username = 'nombre_usuario';
\end{lstlisting}

\subsection{¿Puedo recuperar un artículo eliminado?}

No, la eliminación de artículos es permanente. Se recomienda hacer respaldos regulares de la base de datos.

\subsection{¿Cómo hago un respaldo de la base de datos?}

\textbf{Desde línea de comandos:}
\begin{lstlisting}[language=bash]
mysqldump -u root -p blog_db > backup.sql
\end{lstlisting}

\textbf{Desde phpMyAdmin:}
\begin{enumerate}
    \item Seleccionar la base de datos ``blog\_db''
    \item Ir a la pestaña ``Exportar''
    \item Seleccionar formato SQL
    \item Hacer clic en ``Continuar''
\end{enumerate}

\subsection{¿El sistema soporta múltiples idiomas?}

Actualmente, el sistema está desarrollado únicamente en español.

\subsection{¿Puedo agregar imágenes a los artículos?}

En la versión actual, los artículos solo soportan texto plano. La funcionalidad de imágenes está planificada para futuras versiones.

\subsection{¿Qué navegadores son compatibles?}

El sistema es compatible con las versiones más recientes de Chrome, Firefox, Edge y Safari.

\newpage
\section{Contacto y Soporte}

\subsection{Equipo de Desarrollo}

\begin{tabular}{ll}
\textbf{Nombre:} & Dylan David Silva Orrego \\
\textbf{Código:} & 20242020130 \\
\textbf{Email:} & ddsilvaO@udistrital.edu.co \\
\\
\textbf{Nombre:} & Maria Alejandra Munevar Barrera \\
\textbf{Código:} & 20242020145 \\
\textbf{Email:} & mamunevarb@udistrital.edu.co \\
\end{tabular}

\subsection{Institución}

\textbf{Universidad Distrital Francisco José de Caldas} \\
Facultad de Ingeniería \\
Programa de Ingeniería de Sistemas \\
Bogotá D.C., Colombia

\subsection{Profesora}

\textbf{Lilia Marcela Espinosa Rodríguez} \\
Programación Avanzada \\
Semestre 2025-3

\subsection{Repositorio del Proyecto}

El código fuente del proyecto está disponible en GitHub:

\url{https://github.com/gefidu/ProyectoFinalAvanzada}

\subsection{Documentación Adicional}

Para información técnica detallada sobre principios SOLID, patrones de diseño y arquitectura del sistema, consultar:

\begin{itemize}
    \item \textbf{Principios y Patrones de Diseño:} \texttt{tex archives/Principios/main.tex}
    \item \textbf{Diagramas del Sistema:} \texttt{tex archives/Diagramas/DIAGRAMS.md}
    \item \textbf{Anteproyecto:} \texttt{tex archives/Anteproyecto/anteproyecto.tex}
\end{itemize}

\subsection{Reporte de Errores}

Si encuentra algún error o tiene sugerencias de mejora, puede:
\begin{itemize}
    \item Contactar directamente al equipo de desarrollo
    \item Crear un issue en el repositorio de GitHub
    \item Enviar un correo electrónico a la profesora del curso
\end{itemize}

\vspace{2cm}

\begin{center}
\textit{Gracias por usar Odally | Blog}
\end{center}

\end{document}