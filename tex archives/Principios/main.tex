\documentclass[11pt, a4paper]{article}

\usepackage[utf8]{inputenc}
\usepackage[T1]{fontenc}
\usepackage[spanish]{babel}
\usepackage{graphicx}
\usepackage{geometry}
\geometry{left=2.5cm, right=2.5cm, top=2.5cm, bottom=2.5cm}
\usepackage{hyperref}
\usepackage{parskip}
\usepackage{amsmath} % Paquete matemático
\usepackage{listings} % Para bloques de código
\usepackage{xcolor} % Para colores en el código
\usepackage{caption} % Para personalizar captiones

\usepackage{fontspec}
\setmainfont{Source Sans Pro}
\setsansfont{Source Sans Pro}
\setmonofont{Source Code Pro}[Scale=MatchLowercase]

% Configuración de colores para listings (igual a tu configuración original)
\definecolor{codebg}{rgb}{0.95,0.95,0.95}
\definecolor{codeframe}{rgb}{0.82,0.82,0.82}
\definecolor{keyword}{rgb}{0.0,0.2,0.65}
\definecolor{comment}{rgb}{0.25,0.5,0.35}
\definecolor{string}{rgb}{0.58,0.0,0.05}
\definecolor{linenumber}{rgb}{0.45,0.45,0.45}

\lstdefinestyle{mystyle}{
  backgroundcolor=\color{codebg},
  frame=single,
  rulecolor=\color{codeframe},
  framesep=6pt,
  framerule=0.6pt,
  basicstyle=\ttfamily\small,
  keywordstyle=\color{keyword}\bfseries,
  commentstyle=\itshape\color{comment},
  stringstyle=\color{string},
  numbers=left,
  numberstyle=\tiny\color{linenumber},
  stepnumber=1,
  numbersep=8pt,
  showstringspaces=false,
  breaklines=true,
  postbreak=\mbox{\textcolor{codeframe}{$\hookrightarrow$}\space},
  tabsize=2,
  captionpos=b,
  xleftmargin=6pt,
  xrightmargin=0pt
}

\lstset{style=mystyle, language=Python}


\begin{document}

% Portada con logo (comenta la siguiente línea si no tienes la imagen)
\begin{titlepage}
    \centering
    % Si tienes la imagen UDLogo.png, descomenta la línea siguiente
    \includegraphics[width=6cm]{figures/UDLogo.png}\\[1cm]
    
    {\LARGE Universidad Distrital Francisco José de Caldas\par}
    {\large Facultad de Ingeniería\par}
    {\large Ingeniería de Sistemas\par}
    \vspace{2cm}
    
    {\huge\bfseries Principios de Ingeniería y Patrones de Diseño\par}
    {\huge\bfseries Blog \textit{Odally}\par}
    \vspace{1.5cm}
    
    {\large
    \begin{tabular}{l}
    Alejandra Munevar -- 2024 \\
    Dylan Silva -- 2024 \\
    Moreno Granado Sergio Leonardo -- 20242020091
    \end{tabular}\par}
    \vspace{1.5cm}
    
    {\large \emph{Docente:} Lilia Marcela Espinosa Rodríguez\par}
    \vspace{2cm}
    
    {\large Proyecto Final de Programación Avanzada\par}
    {\large Semestre 2025-3\par}
    \vfill
    
    {\large Diciembre de 2025, Bogotá D.C.\par}
\end{titlepage}

\maketitle
\tableofcontents
\newpage

\section{Introducción}
Este documento es la referencia técnica definitiva sobre las decisiones 
de diseño tomadas en el proyecto. No solo nos adherimos a los principios 
SOLID, sino que aplicamos un espectro más amplio de buenas prácticas 
de ingeniería de software (DRY, KISS, SoC) y patrones de diseño 
clásicos.

\section{Principios SOLID}
\subsection{S - Single Responsibility Principle (SRP)}
\emph{``Una clase debe tener una sola razón para cambiar.''}

\textbf{Aplicación:}
Separamos la persistencia (DAO), la lógica de control (Servlet) y los 
datos (Modelo).

\textbf{Referencia en Código:}
Ver \texttt{com.blog.dao.ConexionBD}.
\begin{lstlisting}
// ConexionBD.java
// RESPONSABILIDAD: Únicamente gestionar conexiones JDBC.
// No valida usuarios, no formatea fechas, solo conecta.
public class ConexionBD {
    // ... lógica de conexión ...
}
\end{lstlisting}

\subsection{O - Open/Closed Principle (OCP)}
\emph{``Abierto a extensión, cerrado a modificación.''}

\textbf{Aplicación:}
El sistema soporta nuevos motores de base de datos sin tocar los 
controladores existentes.

\textbf{Referencia en Código:}
Ver \texttt{com.blog.controller.ArticuloServlet}.
\begin{lstlisting}
// ArticuloServlet.java
// El servlet depende de la abstracción, no de la implementación.
private IArticuloDAO articuloDAO; 
// Si mañana usamos Oracle, cambiamos la inyección aquí, 
// y el resto del código del Servlet permanece intacto.
\end{lstlisting}

\subsection{L - Liskov Substitution Principle (LSP)}
\emph{``Las subclases deben ser sustituibles por sus clases base.''}

\textbf{Aplicación:}
Cualquier implementación de \texttt{IArticuloDAO} funciona en el sistema.

\textbf{Referencia en Código:}
Ver \texttt{com.blog.dao.MySQLArticuloDAO}.
\begin{lstlisting}
// MySQLArticuloDAO.java
// Cumple estrictamente el contrato (interfaz) IArticuloDAO.
// No lanza excepciones inesperadas ni cambia el comportamiento esperado.
public class MySQLArticuloDAO implements IArticuloDAO { ... }
\end{lstlisting}

\subsection{I - Interface Segregation Principle (ISP)}
\emph{``Interfaces específicas para clientes específicos.''}

\textbf{Aplicación:}
Interfaces granulares (\texttt{IArticuloDAO}, \texttt{IUsuarioDAO}) en lugar 
de un ``God DAO''.

\subsection{D - Dependency Inversion Principle (DIP)}
\emph{``Depender de abstracciones, no de concreciones.''}

\textbf{Aplicación:}
Los módulos de alto nivel (Controller) no importan las clases de bajo 
nivel (MySQL Implementation).

\section{Otros Principios de Ingeniería}

\subsection{DRY (Don't Repeat Yourself)}
\emph{``Cada pieza de conocimiento debe tener una representación 
única, inequívoca y autorizada dentro de un sistema.''}

\textbf{Aplicación:}
\begin{itemize}
    \item \textbf{JSTL en Vistas:} Utilizamos JSTL (\texttt{c:forEach}, 
    \texttt{c:url}) para evitar repetir lógica Java (Scriptlets) en 
    múltiples fragmentos de JSP.
    \item \textbf{Singleton de Conexión:} La lógica de leer 
    \texttt{db.properties} y conectar está centralizada en un solo lugar, 
    no copiada en cada DAO.
\end{itemize}

\subsection{KISS (Keep It Simple, Stupid)}
\emph{``La mayoría de los sistemas funcionan mejor si se mantienen 
simples.''}

\textbf{Aplicación:}
Mantenemos los POJOs (\texttt{Articulo.java}) libres de lógica compleja. 
Son estructuras de datos puras.
\begin{lstlisting}
// Articulo.java
// Simpleza absoluta: Campos privados, Getters y Setters.
// Nada de lógica de negocio oscura.
public class Articulo implements Serializable { ... }
\end{lstlisting}

\subsection{SoC (Separation of Concerns)}
\emph{``Separación de preocupaciones.''}

\textbf{Aplicación:}
La arquitectura MVC es la encarnación de este principio.
\begin{itemize}
    \item \textbf{Vista (JSP):} Solo presentación.
    \item \textbf{Controlador (Servlet):} Solo orquestación.
    \item \textbf{Modelo (DAO + Entidades):} Solo datos y negocio.
\end{itemize}

\section{Patrones de Diseño Aplicados}

\subsection{Singleton Pattern}
\textbf{Problema:} Necesitamos coordinar el acceso a un recurso compartido (BD).
\textbf{Solución:} \texttt{ConexionBD} garantiza una única instancia 
global.

\begin{lstlisting}
// ConexionBD.java
public static ConexionBD getInstancia() {
    if (instancia == null) {
        instancia = new ConexionBD();
    }
    return instancia;
}
\end{lstlisting}

\subsection{DAO Pattern (Data Access Object)}
\textbf{Problema:} El código de negocio no debería contener SQL.
\textbf{Solución:} Encapsulamos el acceso a datos en clases dedicadas 
(\texttt{MySQLArticuloDAO}), abstraídas por interfaces.

\subsection{MVC Pattern (Model-View-Controller)}
\textbf{Problema:} Código espagueti (mezcla de HTML y Java).
\textbf{Solución:} Jakarta EE estándar con Servlets (C) y JSPs (V).

\end{document}