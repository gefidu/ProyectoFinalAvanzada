\documentclass[11pt, a4paper]{article}

\usepackage[utf8]{inputenc}
\usepackage[T1]{fontenc}
\usepackage[spanish]{babel}
\usepackage{graphicx}
\usepackage{geometry}
\geometry{left=2.5cm, right=2.5cm, top=2.5cm, bottom=2.5cm}
\usepackage{hyperref}
\usepackage{parskip}

\title{\textbf{Manual de Usuario} \\ \large Sistema de Gestión de Contenidos (Blog) JavaWeb}
\author{Equipo de Desarrollo}
\date{\today}

\begin{document}

\maketitle
\tableofcontents
\newpage

\section{Introducción}
Este manual tiene como objetivo guiar al usuario en el uso del Sistema de Gestión de Contenidos (Blog). La aplicación permite a los usuarios leer artículos publicados y a los administradores gestionar dicho contenido.

\section{Acceso a la Aplicación}
Para acceder al sistema, abra su navegador web favorito e ingrese la siguiente dirección URL:
\begin{center}
    \texttt{http://localhost:8080/AdvancedFinalProject/}
\end{center}

\section{Área Pública}
Esta sección es accesible para cualquier visitante y no requiere autenticación.

\subsection{Ver Lista de Artículos}
Al ingresar a la aplicación, verá la página principal con un listado de los artículos más recientes.
\begin{itemize}
    \item Cada tarjeta muestra el título, un resumen del contenido, el autor y la fecha de publicación.
\end{itemize}

\subsection{Leer un Artículo}
Para leer el contenido completo de un artículo:
\begin{enumerate}
    \item Haga clic en el botón \textbf{"Leer más"} ubicado en la parte inferior de la tarjeta del artículo.
    \item Se abrirá una nueva página con el texto completo.
    \item Para volver, puede usar el botón \textbf{"Volver al inicio"} o el menú de navegación.
\end{enumerate}

\section{Área de Administración}
Esta sección es exclusiva para usuarios autorizados.

\subsection{Iniciar Sesión}
1. En la barra de navegación superior, haga clic en \textbf{"Iniciar Sesión"}.
2. Ingrese su nombre de usuario (ej. \texttt{admin}) y contraseña.
3. Haga clic en el botón \textbf{"Ingresar"}.

\subsection{Dashboard}
Una vez autenticado, será redirigido al Dashboard, donde podrá ver estadísticas rápidas sobre el número de artículos publicados.

\subsection{Gestión de Artículos}
Desde el menú de administración, seleccione \textbf{"Artículos"} para acceder a las opciones de gestión.

\subsubsection{Crear Nuevo Artículo}
1. Haga clic en el botón verde \textbf{"Nuevo Artículo"}.
2. Complete el formulario con el \textbf{Título} y el \textbf{Contenido}.
3. Haga clic en \textbf{"Guardar"}.

\subsubsection{Editar Artículo}
1. En la lista de artículos, identifique el que desea modificar.
2. Haga clic en el botón azul \textbf{"Editar"}.
3. Modifique la información necesaria y haga clic en \textbf{"Actualizar"}.

\subsubsection{Eliminar Artículo}
1. En la lista de artículos, identifique el que desea borrar.
2. Haga clic en el botón rojo \textbf{"Eliminar"}.
3. Confirme la acción si el sistema lo solicita.

\subsection{Cerrar Sesión}
Para salir del sistema de forma segura, haga clic en el botón \textbf{"Cerrar Sesión"} ubicado en la esquina superior derecha de la barra de navegación.

\end{document}
